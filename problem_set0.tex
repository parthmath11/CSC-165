\documentclass[12pt]{article}
\usepackage{amsmath}
\usepackage[margin=2.5cm]{geometry}
\usepackage{csc}

% Document metadata
\title{CSC 165 - Problem Set 0}
\author{By Parth Patel}
\date{\today}


% Document starts here
\begin{document}
\maketitle

\section*{My Courses}
\begin{itemize}
\item CSC 165 -Mathematical Expression and Reasoning for Computer Science - Francois Pitt
\item CSC A48 - Introduction to Computer Science  II - Marzeih Ahmadzadeh
\item MAT 137 - Calculus - Jean Baptiste Campesato
\end{itemize}

\bigskip
\section*{Set Notation}

% Fill in the following:
\[
S_1 \cap S_2 = \{ x \in \R  \>| \> x = 2n\> \> \> \cap \> \> \> 0 < n \leq 7 \> \> \> , \> \> n \in \N\}
\]

\bigskip
\section*{A Truth Table}
\bigskip
\begin{tabular}{c c c c c c c}  % This specifies a table with 3 columns, all centred.

% Table header. Notice the use of the alignment character "&" and newline "\\".
P & Q & R & $\neg$ Q & P $\lor$ $\neg$ Q & P $\Leftrightarrow$ R &(P $\lor$ $\neg$ Q) $\Rightarrow$ (P $\Leftrightarrow$ R ) \\

% Draw a horizontal line
\hline


% The table rows
F & F & F & T & T & T & T \\
F & F & T & T & T & F & F \\
F & T & F & F & F & T & T \\
F & T & T & F & F & F & T \\
T & F & F & T & T & F & F \\
T & F & T & T & T & T & T \\
T & T & F & F & T & F & F \\
T & T & T & F & T & T & T \\
\end{tabular}



% Note: no newline "\\" on the last line of a table.
\newpage
\section*{A calculation}
\begin{align*}
\log_x(3 \sqrt x) &= k \\
\log (3 \sqrt x) / \log (x) &= k \\
\log (3 \sqrt x) &= k \log(x)\\
\log(3 \sqrt x) &= \log (x^k) \\
3 \sqrt x &= x^k \\
9x &= x^{2k} \\
x/x^{2k} &= 9^{-1} \\
x^{1-2k} &= 9^{-1} \\
(1 - 2k) \ln x &= - \ln (9) \\
\ln x &= \frac{-\ln 9}{1-2k} \\
x &= e^{\frac{- \ln 9}{1-2k}} \\
\end{align*}

\end{document}
